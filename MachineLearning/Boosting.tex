%        File: Boosting.tex
%     Created: 一 10 27 09:00 上午 2014 C
% Last Change: 一 10 27 09:00 上午 2014 C
%
\documentclass[a4paper]{article}
\usepackage{fontspec}
\usepackage{tabularx}
\usepackage{enumerate}
\usepackage{amsmath,amssymb}
\usepackage{algorithm}
\usepackage{algorithmic}
\setmainfont{Microsoft YaHei}

\renewcommand{\algorithmicrequire}{\textbf{Initialization:}}
\renewcommand{\algorithmicensure}{\textbf{Output:}}

\XeTeXlinebreaklocale ''zh''
\XeTeXlinebreakskip = 0pt plus 1pt

\begin{document}
\title{Boosting and Application to LTR}
\author{zjgtan}
\date{\today}
\maketitle

\section{What is Boosting?}
\subsection{Additive Model and Boost}

\subsection{Classification and Regression Tree}
\paragraph{Definition}CART是一种典型的Tree-Based模型。Tree-Based Models将input space划分为cuboid regions,并为每一个region分配a simple model(a constance)。Tree-Based Models可以看做是一种model combination method,在input space的每一个点上只有一个model负责预测。

\paragraph{Prediction}对于一个$x$,我们通过遍历CART树,找到$x$所在的cuboid region。在每一个region中,独立的模型进行预测。

\paragraph{Growing a Tree}采用greedy optimization的策略构建CART模型:从根节点开始,在每次迭代中选择某一维变量及其threshold,最优的选择是使得裂变得到的cuboid regions可以为原region中的样本提供最优的预测。

\subsubsection{Pruning}
\paragraph{Method}Pruning主要目的是控制CART的复杂度,防止Overfitting。一种简单的策略是当residual error的减少下降到某一阈值时,CART树的叶节点停止分裂。但是实践表明,刚开始的split残差可能没有得到很好的减少,但是可能多分类几次却出现了较大的优化。因为贪婪的建树策略只是取局部的最优解。\\
因此,通常的:首先建立一个large tree,基于训练样本的数据量决定叶节点的数量;接着,基于平衡预测精度与模型复杂度的策略进行剪枝。
\paragraph{Criterion}$T_0$表示原始树;叶节点有$\tau=1,\dots,|T|$,其中,$|T|$表示所有叶节点的数量;$R_\tau$表示叶节点$\tau$中的样本数量。\\
对于Regression,$R_\tau$中的最优预测为区域中样本target的均值。
\begin{equation}
  y_\tau=\frac{1}{N_\tau}\sum_{x_n \in R_\tau} t_n
\end{equation}
因此,每个区域的residual sum-of-squares为
\begin{equation}
  Q_\tau(T)=\sum_{x_n \in R_\tau}{t_n-y_\tau}^2
\end{equation}

对于Classification,$p_{\tau k}$表示Region$R_\tau$中$k$类样本的比例,通常有两种指标:\\
cross-entropy
\begin{equation}
  Q_\tau(T)=\sum_{k=1}^{K}p_{\tau k}\ln p_{\tau k}
\end{equation}
Gini index
\begin{equation}
  Q_\tau(T)=\sum_{k=1}^{K}p_{\tau k}(1-p_{\tau k})
\end{equation}
当区域中样本类别越集中于某一类时,这两个指标越小,说明划分越好。

\subsection{AdaBoost}
\subsubsection{AdaBoost算法框架}
\paragraph{}
AdaBoost的算法框架如下表所示
\begin{algorithm}[htb]
  \caption{AdaBoost}
  \label{alg:Ada}
  \begin{algorithmic}[1]
	\REQUIRE 将数据权重${w}_n$初始化为${w}_{n}^{(1)}=\frac{1}{N}$,其中$n=1,\dots,M$
  \ENSURE 最终得到预测模型
  \begin{equation}
	\label{1}
	Y_M(x)=sign(\sum_{m=1}^M\alpha_m y_m(x))
  \end{equation}
  \FOR {each $m \in [1,M]$}
  	\STATE Minimizing the weighted error function
	\begin{equation}
	  \label{2}
	  J_m=\sum_{n=1}^N w_{n}^{(m)}I(y_m(x_n)\neq t_n)
	\end{equation}
	\STATE 计算加权平均误差
	\begin{equation}
	  \label{3}
	  \epsilon_m = \frac{\sum_{n=1}^N w_{n}^{(m)}I(y_m(x_n) \neq t_n)}{\sum_{n=1}^N w_{n}^{(m)}}
	\end{equation}
	\STATE 有弱分类器的组合系数
	\begin{equation}
	  \label{4}
	  \alpha_m = \ln{{\frac{1-\epsilon_m}{\epsilon_m}}}
	\end{equation}
	\STATE 更新训练数据权重
	\begin{equation}
	  \label{5}
	  w_{n}^{(m+1)}=w_{n}^{(m)}\exp\{\alpha_m I(y_m(x_n) \neq t_n)\}
	\end{equation}
   \ENDFOR
\end{algorithmic}
\end{algorithm}

\subsubsection{Interpretation: Minimization of an exponential error function}
\paragraph{}
exponential error function:
\begin{equation}
	  \label{6}
  E=\sum_{n=1}^N \exp\{-t_n f_m(x_n)\}
\end{equation}
其中,$f_m(x_n)$定义为一组弱分类器的线性加权
\begin{equation}
	  \label{7}
  f_m(x)=\frac{1}{m}\alpha_l y_l(x)
\end{equation}
$t_n \in \{-1,1\}$,上述的exponential error function中,当$t_n$与$f_m(x))$同号时,error function得到最小值。我们的目标是最小化exponential error function, with respect to $\alpha_l$ and $y_l(x)$

\paragraph{}
继续推导,我们将得到AdaBoost算法框架。我们不进行全局的误差最小化,而是在固定$y_1(x),\dots,y_{m-1}(x)$和$\alpha_1,\dots,\alpha_m-1$的情况下,寻找$\alpha_m$和$y_m(x)$。首先对error function进行分解
\begin{equation}
  \label{form8}
  \begin{split}
  E=\sum_{n=1}^N \exp\{-t_n f_{m-1}(x_n) - \frac{1}{2} t_n \alpha_m y_m(x_n)\}\\
  =\sum_{n=1}^N w_{n}^{(m)} \exp \{-\frac{1}{2} t_n \alpha_m y_m(x_n)\}
\end{split}
\end{equation}

这里是关键的部分。将训练集划分为第m个弱分类器的正确分类集合$\T_m$和误分类集合$\M_m$
\begin{equation}
  \label{form9}
  \begin{split}
	E=e^{-\frac{\alpha_m}{2}}\sum_{n \in T_m}+e^{\frac{\alpha_m}{2}}\sum_{n \in M_m}w_{n}^{(m)} \\
	=(e^{\frac{\alpha_m}{2}} - e^{-\frac{\alpha_m}{2}})\sum_{n=1}^{N}w_{n}^{(m)}I(y_m(x_n) \neq t_n) + e^{-\frac{\alpha_m}{2}}\sum_{n=1}^{N}w_{n}^{(m)}
  \end{split}
\end{equation}

\paragraph{}继续将上式分解,目标是得到\ref{2}-\ref{5}式的结果。首先对$y_m(x_n)$进行最小化,\ref{form9}中第二项为常量,仅有第一项与$y_m(x_n)$有关,并且与AdaBoost算法框架的\ref{2}式相同,因此优化\ref{2}式就是对$y_m(x_n)$进行优化。
\paragraph{}继而对$\alpha_m$进行优化,简单偏导,十分易得。
\paragraph{}最后是对权重进行更新。有公式$t_ny_m(x_n)=1-2I(y_m(x_n)\neq t_n)$容易得到结果。

\section{Gradient Boosting}
\subsection{Numerical Optimization}
\subsection{Gradient Boosting}
\subsection{Application}

\section{Application: Learning to rank}
\subsection{RankBoost}
\subsection{GBRank}


\end{document}


